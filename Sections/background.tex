\section{\uppercase{Related Work}}
\subsection{Brain-computer interface ``in the wild''}

\noindent Wider adoption of BCI systems relies on two main streams of research: (i) the development of ergonomic sensors suitable for use in naturalitic settings and (ii) the ability to adapt lab-developed BCI strategies to the new constraints that these sensors impose on our data processing abilities. 

% TODO: cite surveys of devices-- there's a survey of EEG hardware from wheelchairs from south africa but maybe tehre's something moe recent
% TODO: study comparing consumer grade to medical grade EEG ---- just read one recently, it's in my zotero
Many inexpensive, comfortable EEG devices have come to market, most of which use ``dry'' electrodes that do not require special gels. Compared to their lab-based counerparts, these devices have many fewer electrodes, thus limited spatial resolution, and produce significantly noisier signals \textcolor{red}{\bf (is there a benchmark available in the literature?)}. Regardless, past work has demonstrated several mobile-ready BCI systems that use these scanning devices, and the Neurosky MindSet in particular (the headset used in this study - a single, dry EEG electrode placed roughly at FP2, which connects wirelessly to phones and computers, and sells for roughly 100USD) has been used to succesfully detect emotional states, event-related potentials (ERP), and employ brain-based biometric authentication \cite{crowley_evaluating_2010,grierson_better_2011,adams_i_2013}.  However, the use of consumer EEGs for the direct, real-time control of software interfaces has proven more difficult \cite{carrino_self-paced_2012,larsen_classification_2011}. \textcolor{red}{\bf add a sentence like: While we should expect massive improvement of consumer-grade EEG in the near future, with more sensors (e.g. Muse, Melon), or medical-grade EEG sold as consumer products, we should still expect noisier signal as people will ``wear" and move with there devices. Mobile systems require reducing the bandwidth (currently 1 megaoctet per dry EEG sensor sampling at 512 Hz)}.

{\bf Two main issues : (i) more noise, and (ii) limited bandwidth}

To transition BCI from the lab into naturalistic environments, we must squeeze more signal out of fewer, and less reliable, sensors. Furthermore, since BCIs are envisioned largely as always-available input devices, they will likely be deployed on mobile processors and perhaps even embedded processing systems; ouro computational resources may be more similar to that of a smartphone than of a desktop workstation, and it is feasible that we may need to do some processing ``in the cloud'' (ie., on a more powerful server to which the client sends data over the network, similar to the way Apple's Siri processes voice data). For effective BCI to occur in these environments, we must extract signal in a maximally efficient way so as to limit our computational footprint, and perhaps even to optimize throughput if we wish to ship data to an external server.

\subsection{Statistical signal processing in EEG-based BCI}

\noindent BCI systems generally aim to recognize a user's mental gestures as one of a finite set of discrete symbols, which can be thought of as a pattern recognition task \cite{lotte_review_2007}. The difficulty of this task stems primarily from the variable and non-stationary nature of neural signals: the ``symbols" we wish to identify are expressed differently between individuals, and even vary within individuals from trial to trial \cite{vidaurre_fully_2006,vidaurre_machine-learning-based_2011}. 

In order to compensate for variability in BCI signals, recent work has leveraged adaptive classification algorithms to distinguish between mental gestures \cite{lotte_review_2007,vidaurre_machine-learning-based_2011} \textit{steal some lines introing classificatino algos............maybe steal a line explaining what a classifier is in the context of a BCI system, and how we train one......}.  In classification algorithms generally, larger feature vectors require that an exponential increase in the amount of data needed to describe classes, a property known as ``the curse of dimensionality'' \cite{jain_statistical_2000,raudys_small_1991}. Traditionally, BCI applications rely on dense, high-diemsnional feature vectors produced by multi-electrode scanning caps with high temporal resolution, so dimensionality represents a major bottleneck in training classification algorithms. This bottleneck threatens the responsiveness of BCI from a user experience standpoint and places high requirements on end users' hardware.

\subsection{Online, co-adaptive calibration}

\noindent Learning to control a BCI system involves more than an adaptive software algorithm. Shenoy et al (2006) frame BCI learning as a cooperation between two adaptive systems: the BCI's algorithms and the human user \cite{shenoy_towards_2006}.  By building interfaces in which the user and the BCI ``co-adapt'' during an interactive calibration step, past work has turned BCI novices into competent users over the course of hours instead of days or weeks, and without manual calibration by a researcher \cite{vidaurre_fully_2006,vidaurre_co-adaptive_2011,vidaurre_machine-learning-based_2011}. 
Past work on co-adaptive BCI has used a several-step approach in which the system feeds preprocessed data to an adaptive classifier, which uses new and past data to optimize and recalculate itself, either during intermittent, offline steps or continuously online \cite{vidaurre_fully_2006,shijian_lu_unsupervised_2009,das_unsupervised_2013}. During calibration, users perform ``labeled'' (that is, known) mental gestures in order to produce samples for the classifier. Meanwhile, the classifier performs various experiments in which it attempts to establish which features of the data are most informative. Systems may generate multiple models in parallel and combine their decisions democratically (an ``ensemble approach''). After several calibration steps, the system is able to estimate the user's control by assessing its model's accuracy on samples it has already recorded.

\subsection{Co-adaptive BCI in naturalistic settings}

% thomas's note on "change for the same subject over time": Since you have mentioned this issue in the manuscript, you might want to test robustness of our classifier over two different sessions. If remember well, we have data for this.
\noindent For the control of interface systems, it is crucial that mental gestures be actuated intentionally, and that the system's interpretation be immediately verifiable by the user. \cite{mcfarland_brain-computer_2011} \textit{Maybe a line here about how the system needs to be fast for responsiveness} Efficient calibration is particularly crucial for real-world use, as EEG signals vary between subjects, and could even change within the same subject over time. From a technical standpoint, calibration amounts to the training and re-training of one or several adaptive algorithms. Calibration can be processing-intensive on a mobile device, especially if the system is computing multiple candidate models. This requires a great deal of online signal processing, which entails not only the computational time required to train the classifier but also the space required to handle the data and the time required to read and write the data from memory or from disk.

% drop another hint about embedded sensors and doing stuff in the cloud
