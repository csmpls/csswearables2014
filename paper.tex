\documentclass[a4paper,twoside]{article}

\usepackage{epsfig}
\usepackage{subfigure}
\usepackage{calc}
\usepackage{amssymb}
\usepackage{amstext}
\usepackage{amsmath}
\usepackage{amsthm}
\usepackage{multicol}
\usepackage{pslatex}
\usepackage{apalike}
\usepackage{SCITEPRESS}
\usepackage[small]{caption}
\usepackage{color}

\subfigtopskip=0pt
\subfigcapskip=0pt
\subfigbottomskip=0pt

\begin{document}





\title{Field Recordings, Mind, and the Eternal Hum or Whisper: Observing neural activity in the wild and at scale}

\author{\authorname{Nick Merrill\sup{1}, Thomas Maillart\sup{1} and John Chuang\sup{1}}
\affiliation{\sup{1}School of Information, UC Berkeley, California, USA}
\email{ffff@berkeley.edu}
}

\keywords{?}




\abstract{whysoever why and when, and why not}

\onecolumn \maketitle \normalsize \vfill




\section{Introduction}

Our ability to build effective, useful brain-computer interface (BCI) is limited by the capabilities of our scanning devices and by our inchoate knowledge of neural activity, especially about how it looks outside of lab settings. While we expect mobile brain scanners to improve dramatically over the next few years, our ability to understand brains ``in the wild'' is hampered primarily by a paucity of relevant data. Almost all neurological studies, BCI research included, occurs under controlled conditions rather than in the normal human habitat, a perceptually rich environment in which people are moving freely and interacting with space, objects and people.
% consider adding: (the Melon headband, Interaxon Muse and Emotiv Insight all promise much better scanning resolution, comfort and battery life than current-generation devices)

Presently, there exists no large-scale repository for ``field recordings'' (i.e., long-timescale recordings taken in uncontrolled, non-lab environments) of neural signals, making it difficult for researchers to draw the sorts of observations that might improve BCIs, and our understanding of mind generally.

The difficulty of dealing with subject-to-subject and trial-to-trial changes in neural signals provides a salient example of our knowledge gap: a BCI that works with one person will most likely not work with another, and may even stop working on the original subject over time, due simply to the fact that minds shift and change and grow. It is tempting to believe that such changes, especially those observed within subjects over time, can be anticipated or at least described statistically, to some extent, given enough data. However, without a large-scale source of neural field recordings, it is difficult or impossible for researchers to make the sorts of observations necessary to investigate such patterns.

This paper imagines a platform for neural field recordings, approaching the problem from an information architecture standpoint and from a user interface one. How do we scale a data collection platform to hundreds of thousands of users, all of whom are producing megabytes of data every minute? How do we incentivize users to donate their neural data, and how do we assure their privacy? How might we manage access and ownership in a publically available research repository while maintaining the confidentiality of individuals involved?

*maybe we talk about what this paper covers here, in summary, when we figure that out*






\section{Background \& Related Work}

\subsection{Brain-computer interface}

Brain-computer interface aims to establish a direct connection between the human nervous system and a digital computer, without intermediate muscular intervention. Obvious applications include devices and tools for the severely disabled, for whom muscular activity is difficult or impossible, but BCI has a potentially very wide audience if it is convenient enough to use; it promises to increase our overall communicative capacity with digital machines, and to passively monitor our cognitive load and emotional states, helping us better adapt to our environment and to ourselves.

Attempts at BCI generally employ adaptive, machine-learning algorithms, as neural signals vary from person to person, and change within individuals over time (this is refferred to as ``nonstationarity'' - that is,  that there exists no static model of neural functioning that will work reliably with all people at all times). Some \textit{a priori} neurological knowledge is generally employed, usually around a central strategy (e.g., P300, SSVP, motor imagery, etc; see [] for a review of BCI strategies). Thereafter, an adaptive algorithm, usually a linear classifier, is trained on sample data, and this calibration closes the gap between the theoretical basis behind the system and the observed data in reality.

\subsection{BCI in the wild}

There have been numerous attempts to deploy brain-computer interfaces in outside-of-lab settings. Many wireless EEG headsets exist - there are at least four on the market presently, and three new devices slated for release in the immediate future. From an application standpoint, some commercial ventures into ``direct-control'' BCI applications have recieved wide media attention as of late (e.g. an application for Google Glass that takes a picture when the user focuses). Historically, consumer BCI applications have centered more around affective and emotional measurement such as stress detectors, meditation coaches, applications aimed at monitoring mental workload, and so on. The claims behind these devices are difficult to rigorously validate, as the developers of these applications understandably do not publish detailed information about their use.

From the academic side of things, a few applications have surfaced employing consumer-grade wireless headsets. One application authenticates users using their brainwaves alone, and various others have replicated the levels of control achieved with in-lab direct-control BCI systems, with varying levels of success. A few of these projects have cognitive load measurements in real-life work situations, and a few have looked at in-the-wild emotional response. Overwhelmingly, however, academic approaches to BCI have approached their subjects and analysis from inside the lab, while free-market ventures into BCI have not rigorously collected or disseminated data on their applications' use.


\subsection{Large-scale data repositories for scientific research}








\section{Specifications}

The BCI community, as well as the neuroscience and wearable computing \textit{(have i really established this, does this paper really establish its relevance to the larger wearables community)} communities at large, could benefit from a large-scale repository of neurological data recorded in real-world settings. In this section, we describe the specifications for an open collection platform aimed at field recordings of neural data. We focus specifically on the use of electroencephalography (EEG) \textit{(mention eeg stuff earlier?)} due to its wide adoption in BCI and as a format for consumer brainscanners.

\subsection{Scalability}

A system for storing neural recordings is unlike traditional web applications (Twitter, Facebook, Youtube) in that we expect each user to produce a much larger volume of (raw) data, and to submit that data more frequently. The Neurosky MindWave, a current-generation wireless EEG with a single electrode, produces approximately a megabyte of raw data every minute. This bitrate will increase linearly with the number of electrodes on the device; just the devices on the immediate horizon will produce up to five times this volume of data. With each user producing gigabytes of data every day in chronic recording scenarios, the system will not scale well to large number of users.

\subsection{Redundancy \& Control}

Any useful repository of scientific data must be openly accessible, but ownership over the data itself becomes a more nuanced issue as the volume of data increases. For large data repositories, the cost of maintaining the requisite server space and the bandwidth necessary to make the data on those servers widely accessible can be formidable, at least with traditional, centralized architectures. Besides, these sorts of architectures put data under the ownership of a single entity or individual, which is problematic occasionally from a practical standpoint (data loss, outages, economic issues, etc), but from an intellectual standpoint as well, as scientific data ought to be ..... well. lol

\subsection{Privacy}

In line with the regulations enforced by any internal review board, all data collected on this platform must be anonymous or, at the very least, psuedonymous. Users should be able to remove their data from the system at any time (``the right to be forgotten'' as it has lately been called, analogous to the right of experimental subjects to withdraw from a study), and there might exist an option by which users can require permission for their data to be accessed, or at least to be accessed on a continual basis.

\subsection{End user concerns}

If we are to collect intimate information from volunteers, we must certainly offer them something in return. Any system that collects data on a large scale must be equally capable of delivering value based on that data, preferably such that applications could adjust fluidly to future observations or developments based off the data collected through the system at large. 






\section{Design recommendations}

\subsection{A lossy fileformat for EEG data}
Raw recordings may not be necessary to produce useful insights about neural data, especially at high volume. A lossy file format may be sufficient, the proverbial mp3 of neural data. What might this file format look like?

One option is to ``quantize'' the data - to pull points we believe to be relevant or informative and to discard the rest of the signal. One quantization method we have found particularly effective is the selection of logarithmically-spaced points in a continuous power spectrum of EEG data. We find that even at low numbers of bins (25-50, about 80 bytes), we can achieve very strong classification accuracy on mental imagery tasks using a support vector machine (SVM). Our accuracy can be bulstered by compressing continuous readings in the time dimension by averaging each bin reading over a number of signals, an effect that we have found to persit at EEG recordings up to ten seconds in length. This latter technique has the happy effect of further reducing our data output (at ten seconds, the total bin data would be 80 bytes as opposed to 80 bytes * 20 power spectra). These findings are summarized in Figure 1.

Another approach is to maximize the informativeness of the signal by replacing individual data points with a model that describes statistically how a reading appeared over a period of time. A brain with relatively stable activity (e.g. deep sleep, or REM, typified by elevated activity in the gamma band and supressed activity in the alpha band) need not be described as an hour of continuous readings; instead, it could be described as a statistical range of observed values, or even as a cyclical pattern that repeats, if such a model is relevant and appropriate. Data in this format need be represented merely by the model and whatever observed data deviated from it (e.g., ``an hour of REM with a spike in alpha activity between seconds 530 and 620''). By representing readings this way, an EEG signal could be described probabilistically at any point in time with representing discrete readings from the device. \textit{does this sort of technique have a name? i know this has to do with exploiting low entropy in the signal, so i feel it must have some kind of information theoretic way of being expressed or quantified; i wouldn't be surprised if there's a class of compression algos aimed at generating patterns like these.}

\subsection{A decentralized network}

\subsection{A decoupled application layer}





\section{Discussion}




% \section{\uppercase{Introduction}}
\label{sec:introduction}

\noindent Brain-computer interface (BCI) systems establish a direct communcative link between the brain and an electronic system \cite{dornhege_toward_2007,mcfarland_brain-computer_2011}.  Recently, the combination of machine learning algorithms and non-invasive electroencephelographs (EEG) has yielded proof-of-concept systems ranging from brain-controlled keyboards and wheelchairs to prosthetic arms and hands \cite{blankertz_note_2007,millan_combining_2010,d._mattia_brain_2011,hill_practical_2014,campbell_neurophone:_2010}. 
% TODO: more cites on cool EEG applications?

There are a few reasons why these BCI systems have not found wide adoption outside of lab settings. For one, they require large, complex scanning caps, which are impractical for disabled users and generally undesirable for ergonomic reasons \cite{ekandem_evaluating_2012,leeb_transferring_2013}. Meanwhile, the amount of data produced by these caps is large, which places high requirements on end-user's hardware. Finally, BCIs often require upward of an hour to calibrate to their users, and may require regular recalibration due to the nonstationary nature of EEG signals. \cite{vidaurre_fully_2006,vidaurre_co-adaptive_2011,blankertz_non-invasive_2007} 

For BCI systems to enjoy wider adoption ``in the wild,'' they must calibrate to individual users quickly and achieve decent information transfer rates (ITR), but with fewer sensors than their lab-based counterparts, and with noisier signals, as data acquisition will occur while people are performing daily tasks, moving, walking, talking, and so on. As an added challenge, their computational firepower may be limited by the mobile \& wearable computing architectures on which they will most likely be deployed. 

Do we really need dense, high-dimensional EEG data to achieve acceptable accuracy in BCI systems? In this study, we use recordings from a consumer-grade single, dry electroencephalographic (EEG) sensor to simulate the calibration of a simple BCI, and investigate the effect of a novel signal extraction technique on the system’s computational performance and accuracy.First, we find that our signal extraction technique significantly increases the computational speed of a classification-based BCI without a significant detriment to accuracy. Second, we find evidence that this technique can be used to build effective mental task classifiers with under a minute of training data.
% \section{\uppercase{Related Work}}
\subsection{Brain-computer interface ``in the wild''}

\noindent Wider adoption of BCI systems relies on two main streams of research: (i) the development of ergonomic sensors suitable for use in naturalitic settings and (ii) the ability to adapt lab-developed BCI strategies to the new constraints that these sensors impose on our data processing abilities. 

% TODO: cite surveys of devices-- there's a survey of EEG hardware from wheelchairs from south africa but maybe tehre's something moe recent
% TODO: study comparing consumer grade to medical grade EEG ---- just read one recently, it's in my zotero
Many inexpensive, comfortable EEG devices have come to market, most of which use ``dry'' electrodes that do not require special gels. Compared to their lab-based counerparts, these devices have many fewer electrodes, thus limited spatial resolution, and produce significantly noisier signals \textcolor{red}{\bf (is there a benchmark available in the literature?)}. Regardless, past work has demonstrated several mobile-ready BCI systems that use these scanning devices, and the Neurosky MindSet in particular (the headset used in this study - a single, dry EEG electrode placed roughly at FP2, which connects wirelessly to phones and computers, and sells for roughly 100USD) has been used to succesfully detect emotional states, event-related potentials (ERP), and employ brain-based biometric authentication \cite{crowley_evaluating_2010,grierson_better_2011,adams_i_2013}.  However, the use of consumer EEGs for the direct, real-time control of software interfaces has proven more difficult \cite{carrino_self-paced_2012,larsen_classification_2011}. We expect significant improvements from consumer-grade EEG devices in the near future, with more sensors and better signal quality (e.g. Interaxon Muse, Melon headband, Emotiv Insight); however, we expect the signal from these devices will remain noisier than lab-basde counterparts, as people will be wearing and using them while moving, and in uncontrolled environments with ambient electromagnetic signals interfering with endogenous biosignals. \textcolor{red}{\bf Mobile systems require reducing the bandwidth (currently 1 megaoctet per dry EEG sensor sampling at 512 Hz) <-- why/what does this mean? }.

{\bf Two main issues : (i) more noise, and (ii) limited bandwidth}

To transition BCI from the lab into naturalistic environments, we must squeeze more signal out of fewer, and less reliable, sensors. Furthermore, since BCIs are envisioned largely as always-available input devices, they will likely be deployed on mobile processors and perhaps even embedded processing systems; ouro computational resources may be more similar to that of a smartphone than of a desktop workstation, and it is feasible that we may need to do some processing ``in the cloud'' (ie., on a more powerful server to which the client sends data over the network, similar to the way Apple's Siri processes voice data). For effective BCI to occur in these environments, we must extract signal in a maximally efficient way so as to limit our computational footprint, and perhaps even to optimize throughput if we wish to ship data to an external server.

\subsection{Statistical signal processing in EEG-based BCI}

\noindent BCI systems generally aim to recognize a user's mental gestures as one of a finite set of discrete symbols, which can be thought of as a pattern recognition task \cite{lotte_review_2007}. The difficulty of this task stems primarily from the variable and non-stationary nature of neural signals: the ``symbols" we wish to identify are expressed differently between individuals, and even vary within individuals from trial to trial \cite{vidaurre_fully_2006,vidaurre_machine-learning-based_2011}. 

In order to compensate for variability in BCI signals, recent work has leveraged adaptive classification algorithms to distinguish between mental gestures \cite{lotte_review_2007,vidaurre_machine-learning-based_2011} \textit{steal some lines introing classificatino algos............maybe steal a line explaining what a classifier is in the context of a BCI system, and how we train one......}.  In classification algorithms generally, larger feature vectors require that an exponential increase in the amount of data needed to describe classes, a property known as ``the curse of dimensionality'' \cite{jain_statistical_2000,raudys_small_1991}. Traditionally, BCI applications rely on dense, high-diemsnional feature vectors produced by multi-electrode scanning caps with high temporal resolution, so dimensionality represents a major bottleneck in training classification algorithms. This bottleneck threatens the responsiveness of BCI from a user experience standpoint and places high requirements on end users' hardware.

\subsection{Online, co-adaptive calibration}

\noindent Learning to control a BCI system involves more than an adaptive software algorithm. Shenoy et al (2006) frame BCI learning as a cooperation between two adaptive systems: the BCI's algorithms and the human user \cite{shenoy_towards_2006}.  By building interfaces in which the user and the BCI ``co-adapt'' during an interactive calibration step, past work has turned BCI novices into competent users over the course of hours instead of days or weeks, and without manual calibration by a researcher \cite{vidaurre_fully_2006,vidaurre_co-adaptive_2011,vidaurre_machine-learning-based_2011}. 
Past work on co-adaptive BCI has used a several-step approach in which the system feeds preprocessed data to an adaptive classifier, which uses new and past data to optimize and recalculate itself, either during intermittent, offline steps or continuously online \cite{vidaurre_fully_2006,shijian_lu_unsupervised_2009,das_unsupervised_2013}. During calibration, users perform ``labeled'' (that is, known) mental gestures in order to produce samples for the classifier. Meanwhile, the classifier performs various experiments in which it attempts to establish which features of the data are most informative. Systems may generate multiple models in parallel and combine their decisions democratically (an ``ensemble approach''). After several calibration steps, the system is able to estimate the user's control by assessing its model's accuracy on samples it has already recorded.

\subsection{Co-adaptive BCI in naturalistic settings}

% thomas's note on "change for the same subject over time": Since you have mentioned this issue in the manuscript, you might want to test robustness of our classifier over two different sessions. If remember well, we have data for this.
\noindent For the control of interface systems, it is crucial that mental gestures be actuated intentionally, and that the system's interpretation be immediately verifiable by the user. \cite{mcfarland_brain-computer_2011} \textit{Maybe a line here about how the system needs to be fast for responsiveness} Efficient calibration is particularly crucial for real-world use, as EEG signals vary between subjects, and could even change within the same subject over time. From a technical standpoint, calibration amounts to the training and re-training of one or several adaptive algorithms. Calibration can be processing-intensive on a mobile device, especially if the system is computing multiple candidate models. This requires a great deal of online signal processing, which entails not only the computational time required to train the classifier but also the space required to handle the data and the time required to read and write the data from memory or from disk.

% drop another hint about embedded sensors and doing stuff in the cloud


\vfill
\bibliographystyle{apalike}
{\small
\bibliography{references}}

\vfill
\end{document}

