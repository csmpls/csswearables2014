\documentclass[a4paper,twoside]{article}

\usepackage{epsfig}
\usepackage{subfigure}
\usepackage{calc}
\usepackage{amssymb}
\usepackage{amstext}
\usepackage{amsmath}
\usepackage{amsthm}
\usepackage{multicol}
\usepackage{pslatex}
\usepackage{apalike}
\usepackage{SCITEPRESS}
\usepackage[small]{caption}
\usepackage{color}

\subfigtopskip=0pt
\subfigcapskip=0pt
\subfigbottomskip=0pt

\begin{document}





\title{Field Recordings, Mind, and the Eternal Hum or Whisper}

\author{\authorname{Nick Merrill\sup{1} and John Chuang\sup{1}}
\affiliation{\sup{1}School of Information, UC Berkeley, California, USA}
\email{ffff@berkeley.edu}
}

\keywords{?}




\abstract{whysoever why and when, and why not}

\onecolumn \maketitle \normalsize \vfill




\section{Introduction}

Our ability to build effective, useful brain-computer interface (BCI) is limited by the capabilities of our scanning devices and by our inchoate knowledge of neural activity, especially as it exists outside of lab settings. While we expect mobile brain scanners to improve dramatically over the next few years, our ability to understand brains ``in the wild'' is limited primarily by a paucity of relevant data. Almost all neurological studies, BCI research included, occurs under controlled conditions rather than in the normal human habitat, a perceptually rich environment in which people are moving freely and interacting space, objects and people.
% consider adding: (the Melon headband, Interaxon Muse and Emotiv Insight all promise much better scanning resolution, comfort and battery life than current-generation devices)

Presently, there exists no large-scale repository for ``field recordings'' (i.e., long-timescale recordings taken in uncontrolled, non-lab environments) of neural signals, making it difficult for researchers to draw the sorts of observations that might improve BCIs, and our understanding of mind generally.

The difficulty of dealing with subject-to-subject and trial-to-trial changes in neural signals provides a salient example of our knowledge gap: a BCI that works with one person will most likely not work with another, and may even stop working on the original subject over time, due simply to the fact that minds shift and change and grow. It is tempting to believe that such changes, especially those observed within subjects over time, can be predicted statistically, at least to some extent, given enough data. However, without a large-scale source of neural field recordings, it is difficult or impossible for researchers to make the sorts of observations necessary to investigate such patterns.

This paper imagines a platform for neural field recordings, approaching the problem from an information architecture standpoint and from a user interface one. How do we scale a data collection platform to hundreds of thousands of users, all of whom are producing megabytes of data every minute? How do we incentivize users to donate their neural data, and how do we assure their privacy? How might we manage access and ownership in a publically available research repository while maintaining the confidentiality of individuals involved?

*maybe we talk about what this paper covers here, in summary, when we figure that out*






\section{Background \& Related Work}

- everyman's intro to bci, w ML esp

- BCI in the wild (still so bad)

- data repositories for science








% \input{Sections/intro}
% \section{\uppercase{Related Work}}

\noindent 


\vfill
\bibliographystyle{apalike}
{\small
\bibliography{references}}

\vfill
\end{document}

